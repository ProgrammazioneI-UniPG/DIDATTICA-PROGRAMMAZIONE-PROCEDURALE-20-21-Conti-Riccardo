\documentclass[addpoints,11pt]{exam}
\usepackage[top=0.5in, bottom=0.5in, left=0.5in, right=0.5in]{geometry}
\usepackage[utf8]{inputenc}
\usepackage{listings}
\usepackage{color,graphicx}
\usepackage{multicol}
\usepackage{MnSymbol}



\definecolor{codegreen}{rgb}{0,0,0}
\definecolor{codegray}{rgb}{0,0,0}
\definecolor{codepurple}{rgb}{0,0,0}
\definecolor{backcolour}{rgb}{1,1,1}

\lstdefinestyle{mystyle}{
	backgroundcolor=\color{backcolour},   
	commentstyle=\color{codegreen},
	keywordstyle=\color{black},
	numberstyle=\tiny\color{codegray},
	stringstyle=\color{codepurple},
	basicstyle=\footnotesize,
	breakatwhitespace=false,         
	breaklines=true,                 
	captionpos=b,                    
	keepspaces=true,                 
	numbers=left,                    
	numbersep=5pt,                  
	showspaces=false,                
	showstringspaces=false,
	showtabs=false,                  
	tabsize=2
}

\lstset{style=mystyle}

\pagestyle{empty}

\begin{document}
	\boxedpoints
	\pointname{~punti}
	\begin{center}
		\fbox{\fbox{\parbox{5.5in}{\centering
					Prova scritta Programmazione Procedurale}}}
	\end{center}
	
	\vspace{5mm}
	
	\noindent\makebox[\textwidth]{Nome e Cognome: \rule{8cm}{.1pt} \hspace{1cm} Matricola:  \rule{5cm}{.1pt}}
	
	%\vspace{5mm}
	%\makebox[\textwidth]{Matricola:\enspace\hrulefill}
	
	
	\begin{questions} 
		
		\question[1]
		Riportare le conversioni di tipo \underline{implicite} e scrivere quanto valgono alla fine le variabili $a$, $b$, $c$.
		
		\begin{minipage}[t]{0.5\linewidth}
			\begin{lstlisting}[language=C]
				char a= (char) 2, b= (char) 3, c= (char) 4;
				a= 2 * a * b / c;
				if ( a < b ) 
				b = c;
			\end{lstlisting}
		\end{minipage}
		\begin{minipage}[t]{0.5\linewidth}
			\makeemptybox{50pt}
		\end{minipage}
		
		
		
		\question[2]
		Riportare le conversioni di tipo \underline{implicite} e scrivere quanto valgono alla fine le variabili $a$, $b$, $c$.
		
		\begin{minipage}[t]{0.5\linewidth}
			\begin{lstlisting}[language=C]
				unsigned int a = 5U;
				char b = (char) 2;
				float c = 2.0;
				c = (float) a / b;
				b = a / b;
				a = c - b;
			\end{lstlisting}
		\end{minipage}
		\begin{minipage}[t]{0.5\linewidth}
			\makeemptybox{70pt}
		\end{minipage}
		
		
		
		\question[3]
		Riportare le conversioni di tipo \underline{implicite} e scrivere quanto valgono alla fine le variabili $a$, $b$, $c$.
		
		\begin{minipage}[t]{0.5\linewidth}
			\begin{lstlisting}[language=C]
				float a = 2.5;
				long b = 2L;
				int c = 2U;
				char d = (char) 2.5;
				char *p = &d;
				void *v = p;
				if ( c < (int) a )
				b = b*a;
				else
				b = a * d / c;
				a = c * d;
				c = b / a;
				d = a + b + c + d;
				
			\end{lstlisting}
		\end{minipage}
		\begin{minipage}[t]{0.5\linewidth}
			\makeemptybox{145pt}
		\end{minipage}
		
		
		
		\question[1]
		Si fornisca un esempio di effetto collaterale in C, evidenziando i \emph{sequence points} e spiegando perché genera o non genera un warning \emph{multiple unsequenced modifications}, fornendo un controesempio.
		\makeemptybox{75pt}
		
		
		
		\question[2]
		Si definisca una funzione \emph{centro} che ha come parametro un array di \emph{int} e ritorna la media tra il valore massimo e il valore minimo tra gli $n$ elementi presenti nell'array.
		\makeemptybox{100pt}
		
		
		
		\question[2]
		Scrivere una funzione che prenda in input una stringa di lunghezza massima 30 e la stampi invertita.
		\makeemptybox{100pt}
		
		
		
		\question[1]
		Scrivere una funzione di inserimento in in coda ad una lista in maniera \underline{iterativa}.
		\makeemptybox{100pt}
		
		
		
		\question[2]
		Scrivere una funzione di inserimento in coda ad una lista in maniera \underline{ricorsiva}.
		\makeemptybox{100pt}
		
		
		
		\question[4]
		Data la seguente struttura, definire una funzione di nome \emph{positivi} che rimuove gli elementi il cui campo \emph{val} ha valore \underline{minore di 0}.
		
		\begin{minipage}{0.3\linewidth}
			\begin{lstlisting}[language=C]
				struct Lista {
					int val;
					struct Lista* pNext;
				}
			\end{lstlisting}
		\end{minipage}
		\begin{minipage}{0.7\linewidth}
			\makeemptybox{100pt}
		\end{minipage}
		
		
		
		\question[1]
		Indicare quali tra i seguenti sono l-value, dati \emph{int s[3]; int *x = s, *y = x + 3.}
		
		\begin{oneparcheckboxes}
			\choice x + 3;
			\choice *( y + 3 ) = 3;
			\choice *x = y;
			\choice \&y;
			\choice a[3] - 3;
			\choice y = \&x.
		\end{oneparcheckboxes}
		
		
		
		\question [1]Si scriva un programma che permetta di scambiare il valori di due variabili x e y integrando l'utilizzo di puntatori.
		\makeemptybox{100pt}
		
		
		
		\question[2] Si scriva una funzione \emph{matrmalloc}, in grado di allocare dinamicamente una matrice rettangolare di float, le cui dimensioni sono ricevute come parametri. Inizializzare la matrice azzerando tutte le celle.
		\makeemptybox{100pt}
		
		
		
		\question[4] Si scriva una funzione \emph{matrptr} che riceva come parametri dimensioni e puntatore a una matrice rettangolare di float. Inizializzare la matrice azzerando tutte le celle. Spiegare le differenze con l'esercizio precedente evindenziando pro e contro.
		\makeemptybox{125pt}
		
		
		
		\question[4]
		Scrivere cosa stampa la seguente porzione di codice sapendo che $y$ si trova all'indirizzo 0xf0ff010.
		
		\begin{minipage}[t]{0.5\linewidth}
			\begin{lstlisting}[language=C]
				int x= 0xae, a = 12, *y= &x;
				for (int i= 0; i<x ; i++) {
					x = x/2 - (--a);
					printf("%d %d\n", x, --a);
					if (2*a>x) break;
				}
				printf("%d %p\n", x, ((long*) y) + a);
			\end{lstlisting}
		\end{minipage}
		\begin{minipage}[t]{0.5\linewidth}
			\makeemptybox{100pt}
		\end{minipage}
		
		
		
		\question[4]
		Dato \underline{\emph{int s[3]= $\{$511, -666, INT\_MAX)$\}$;}}\\
		\underline{\emph{int  $^*$x = (int$^*$) s;
				char $^*$y= (char$^*$) s;
				x[1]= INT\_MIN - 3; y[2] = $\sim$y[2] - 64;}}  
		sapendo che i due tipi occupano $4$ e $1$ byte, con valori rappresentati in \emph{little endian} e complemento a due, scrivere la mappa di memoria. L'operatore $\sim$ è la negazione bit a bit.
	\end{questions}
	
\end{document}
